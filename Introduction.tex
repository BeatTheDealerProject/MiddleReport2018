\section{背景}
この章では、ブラックジャックの歴史やそれについて今まで研究されてきた戦略、またそれらの戦略の問題点を述べる。
\bunseki{※菱田美紗紀}

\subsection{ブラックジャックの概要と歴史}
ブラックジャックのルーツは1570年にさかのぼる。このころはまだ「ブラックジャック」という名称は使われていなかった。1875年に出版された「The American Hoyle of 1875」という書籍で「ブラックジャック」として紹介されたのが初出である。このゲームが考え出された当初は金銭を賭けることはなく、身内内で楽しむだけのゲームであった。
はじめて金銭が賭けられるようになったのは1910年頃のアメリカ、インディアナ州であるといわれている。当時は競馬以外のギャンブルが違法であり、ブラックジャックも違法カジノでのみ行われていた。その後インディアナ州以外のアメリカ全土に広がっていった。そして現在世界中のカジノで合法的に楽しまれるポピュラーなゲームになった。
\bunseki{※菱田美紗紀}

\subsection{ブラックジャックの戦略の歴史}
ブラックジャックについては多くの人間が戦略を考え出しては、カジノ側に禁止をされてきた歴史がある。1950年に、メリーランド州のとある米国陸軍の研究所に所属していた、Roger Nash Baldwin氏らが研究し始めたのが、ブラックジャックの戦略の研究のはじまりであるといわれている。その後パソコンが発明されたことにより、シミュレーションが容易になったことでさらに戦略の研究は進んでいった。
ブラックジャックには主に有名な戦略が2つ存在する。ベーシックストラテジーと呼ばれる、ディーラーのアップカードと自分の手札によって戦略を決定するものと、カウンティングと呼ばれる、今まで出たカードを記憶して戦略を決定するものである。
まず、ベーシックストラテジーについて詳しく説明する。この戦略は、1962年、カリフォルニア大学アーバイン校の教授である、Edward Thorp氏が、書籍「Beat The Dealer:A Winning Strategy for the Game of Twenty One」にて発表した。この戦略は、ディーラーのアップカード、自分の初期の手札の2つの情報から、次に自分がどのような行動を行えばよいのか決定される戦略である。この戦略を使用すれば勝率は4割~5割弱を出すことができる。
カウンティングについては後期に詳しく調査する予定である。
なお、いずれの戦略もカジノでは禁止されており、これらの戦略を使用していると気づかれたとき、プレイヤーはカジノから追放される。
\bunseki{※菱田美紗紀}

\subsection{従来の戦略の問題点}
まず、ベーシックストラテジーの問題点について述べる。この戦略はデック数が無限個を想定しており、すでに引いたカードの種類や枚数を考慮していない点である。
まず、基本的にカジノで行われるゲームではデック数は有限である。デック数が有限であるということは、以前に引いたカードの種類によって、次に引くカードの確率が変化していくということである。例えば、プレイヤーが2人、デック数が1のゲームをしていると仮定する。ディーラーのアップカードがハートのエース、プレイヤー1の初期の手札がダイヤのエースとクローバーのエース、プレイヤー2の初期の手札がスペードのエースとハートの2が配られたとする。この時点でデック内のエースのカードはすべて配られてしまい、以後のゲームでエースが出る確率は0になっている。しかし、デック数を無限に設定すると、以後のゲームでもエースは無限に存在することになり、確率は4/52になってしまう。これが、デック数無限のシミュレーションと実際のゲームの相違点である。このような相違点を考慮しないため、実際の勝率とは異なってしまい、利益についてもばらつきが出てしまうのである。
またこの戦略は勝率が4割程度しかなく、勝利数だけで言えばカジノ側に負けてしまう。
カウンティングの問題点については後期に詳しく調査する予定である。
\bunseki{※菱田美紗紀}
