\section{ブラックジャックの学習}
本プロジェクトで題材としているブラックジャックというゲームを
実際のプレイも交えて学習した。前章までで説明したベーシックストラテジーを
試すということもその中で行ったが、実際に使用してみると表を覚え、かつ
素早いゲームの進行に合わせながら実行するのは容易ではないという実感が得られた.
もちろんディーラー側に戦略の実行をさとられないようにするためには、ゲームの進行を
止めるなど違和感を持たせるような行動はできるだけ無くす必要がある。以上のことから、
戦略の単純化の必要性を再確認した。
\bunseki{※米村祥裕}

\section{シミュレータの作成}
シミュレータはPython3系で作成を行った。前期までで得られたブラックジャックの戦略比較に用いた
数値はこのシミュレータによって得られた。また開発の効率化のためにバージョン管理システムである
Gitを導入し、Gitによるバージョン管理について学習した。
\bunseki{※米村祥裕}

\section{統計学の学習}
シミュレータの正しさ、シミュレーション結果の分析のために統計学を学習した。
\bunseki{※米村祥裕}

\section{複雑性の学習}
本プロジェクトにおいて、戦略の複雑性を評価することはとても重要な事項である。
複雑性の定義付けのためにコルモゴロフの複雑性を参考にし、コルモゴロフ複雑性の
定義と使われ方について調査し学習した。
\bunseki{※米村祥裕}

\section{ニューラルネットワーク}
本プロジェクトでは最適な戦略の探索を行うための技術の一つとしてニューラルネットワークを
挙げ、勉強会を行った。
\bunseki{※米村祥裕}