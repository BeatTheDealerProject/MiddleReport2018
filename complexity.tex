\section{複雑性を考慮した性能比較とその結果}

本項では、複雑性を考慮した性能比較について、また、その結果について説明する。

\subsection{各戦略の性能評価}

今回はシミュレーションから得た各戦略の勝率と定義した複雑性を用いて、各戦略の性能比較を行った。
性能の基準は以下の二通りを用意した。\\
~~ 1. (勝率) ÷ (複雑性)\\
~~ 2. (勝率) - (複雑性)\\
この評価基準に従って、1デックの時と無限デックの時の性能を表にした。



%\begin{figure}[htbp]
%\begin{center}
%\includegraphics[width=15cm,bb=0 0 602 279]{3.png}
%\end{center}
%\caption{1デックの時の各戦略の性能}
%\label{picture}
%\end{figure}

\begin{table}[H]
\caption{1デックの時の各戦略の性能}
\label{table:data_type}
\begin{center}
\begin{tabular}{|c|c|c|c|c|c|}
\hline
戦略           & 圧縮長 & 勝率    & 複雑性   & 性能1  & 性能2   \\ \hline
ベーシックストラテジー         & 30  & 0.431 & 0.375 & 1.14 & 0.052 \\ \hline
ベーシックストラテジー改変1      & 28  & 0.429 & 0.35  & 1.22 & 0.076 \\ \hline
ベーシックストラテジー改変2      & 26  & 0.430 & 0.325 & 1.21 & 0.072 \\ \hline
15以上になるまでヒットする戦略 & 16  & 0.421 & 0.200 & 2.12 & 0.224 \\ \hline
16以上になるまでヒットする戦略 & 16  & 0.416 & 0.200 & 2.07 & 0.214 \\ \hline
17以上になるまでヒットする戦略 & 16  & 0.410 & 0.200 & 2.05 & 0.209 \\ \hline
18以上になるまでヒットする戦略 & 16  & 0.421 & 0.200 & 1.97 & 0.193 \\ \hline
\end{tabular}
\end{center}
\end{table}


%\begin{figure}[htbp]
%\begin{center}
%\includegraphics[width=15cm,bb=0 0 541 255]{4_.png}
%\end{center}
%\caption{無限デックの時の各戦略の性能}
%\label{picture}
%\end{figure}

\begin{table}[H]
\caption{1デックの時の各戦略の性能}
\label{table:data_type}
\begin{center}
\begin{tabular}{|c|c|c|c|c|c|}
\hline
戦略           & 圧縮長 & 勝率    & 複雑性   & 性能1  & 性能2   \\ \hline
ベーシックストラテジー         & 30  & 0.427 & 0.375 & 1.14 & 0.052 \\ \hline
ベーシックストラテジー改変1      & 28  & 0.424 & 0.35  & 2.12 & 0.224 \\ \hline
ベーシックストラテジー改変2      & 26  & 0.414 & 0.325 & 1.07 & 0.214 \\ \hline
15以上になるまでヒットする戦略 & 16  & 0.424 & 0.200 & 2.05 & 0.209 \\ \hline
16以上になるまでヒットする戦略 & 16  & 0.414 & 0.200 & 1.97 & 0.193 \\ \hline
17以上になるまでヒットする戦略 & 16  & 0.409 & 0.200 & 1.21 & 0.076 \\ \hline
18以上になるまでヒットする戦略 & 16  & 0.393 & 0.200 & 1.21 & 0.072 \\ \hline
\end{tabular}
\end{center}
\end{table}

各戦略を比較し、次のような結果を得た。
まず、勝率のみを考慮した場合、一定の数字以上でスタンドする戦略よりも、ベーシックストラテジーとそれを改変した戦略の方が有意に高い勝率だった。
また、ベーシックストラテジーと改変1、改変2のそれぞれの戦略間には有意な差が見られなかった。
複雑性を考慮して性能を評価した場合、基準値を15に設定した戦略が一番優秀であった。

\bunseki{※渡邊凛}

\subsection{考察}

勝率のみを見ると、1デック、無限デック共にベーシックストラテジーが最も勝率が高かった。しかし、扱いやすさも含めた性能を評価すると、必ずしもベーシックストラテジーが扱いやすいとは限らず、改善の余地があるということが分かった。

\bunseki{※渡邊凛}

\subsection{今後の課題}

前期のプロジェクト学習におけるブラックジャックの前提では、それぞれのゲームは1ゲームで行われており、過去に出たカードが次以降のゲームに影響を与えることはなかった。
そのため、有限のデックで連続したゲームを行った場合の戦略について考える必要がある。

また、今回は勝率のみを考えた場合を想定していた。実際のゲームでは、賭け金の概念があるので、それを導入した場合にどのように利得をプラスにするか、そのための戦略を
考える必要がある。それに伴い、今回のプロジェクトでは省いたダブルダウン、スプリット、サレンダー等のルールを含めて最終的な利得をプラスにする戦略を考えたい。

戦略の扱いやすさについて、今回は複雑性の設定を手動で行い、検証する時間もあまり取らなかったので、評価基準が正確ではない可能性がある。
今後、この評価基準をどのように調整するかも検討の余地がある。

\bunseki{※渡邊凛}

\section{検証結果のまとめ}
これまで行った検証や性能評価による結果をまとめる。\\
勝率のみを考慮した場合
\begin{itemize}
\item 結果1:一定の数字以上でヒットする戦略よりも、ベーシックストラテジーとベーシックストラテジー改変のほうが有意に高い勝率だった
\item 結果2:ベーシックストラテジーベーシックストラテジー改変1、ベーシックストラテジー改変2のそれぞれの戦略間に有意な差はみられなかった
\item 結果3:ベーシックストラテジー改変1と18以上までヒットする戦略にはデック数無限とデック数1で勝率に有意な差があった
\end{itemize}
複雑性を考慮して性能を評価し場合
\begin{itemize}
\item 結果4:15以上になるまでヒットする戦略が1番優秀であることが判明した
\end{itemize}
複雑性を考慮すると、ベーシックストラテジーには改善の余地があることが判明した。