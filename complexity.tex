\section{複雑性を考慮した性能比較とその結果}

本項では、複雑性を考慮した性能比較について、また、その結果について説明する。



\subsection{考察}

勝率のみを見ると、1デック、無限デック共にベーシックストラテジーが最も勝率が高かった。しかし、扱いやすさも含めた性能を評価すると、必ずしもベーシックストラテジーが扱いやすいとは限らず、改善の余地があるということが分かった。

\bunseki{※渡邊凛}

\subsection{今後の課題}

前期のプロジェクト学習におけるブラックジャックの前提では、それぞれのゲームは1ゲームで行われており、過去に出たカードが次以降のゲームに影響を与えることはなかった。
そのため、有限のデックで連続したゲームを行った場合の戦略について考える必要がある。

また、今回は勝率のみを考えた場合を想定していた。実際のゲームでは、賭け金の概念があるので、それを導入した場合にどのように利得をプラスにするか、そのための戦略を
考える必要がある。それに伴い、今回のプロジェクトでは省いたダブルダウン、スプリット、サレンダー等のルールを含めて最終的な利得をプラスにする戦略を考えたい。

戦略の扱いやすさについて、今回は複雑性の設定を手動で行い、検証する時間もあまり取らなかったので、評価基準が正確ではない可能性がある。
今後、この評価基準をどのように調整するかも検討の余地がある。

\bunseki{※渡邊凛}

\section{検証結果のまとめ}
これまで行った検証や性能評価による結果をまとめる。\\
勝率のみを考慮した場合
\begin{itemize}
\item 結果1:一定の数字以上でヒットする戦略よりも、ベーシックストラテジーとベーシックストラテジー改変のほうが有意に高い勝率だった
\item 結果2:ベーシックストラテジーベーシックストラテジー改変1、ベーシックストラテジー改変2のそれぞれの戦略間に有意な差はみられなかった
\item 結果3:ベーシックストラテジー改変1と18以上までヒットする戦略にはデック数無限とデック数1で勝率に有意な差があった
\end{itemize}
複雑性を考慮して性能を評価し場合
\begin{itemize}
\item 結果4:15以上になるまでヒットする戦略が1番優秀であることが判明した
\end{itemize}
複雑性を考慮すると、ベーシックストラテジーには改善の余地があることが判明した。