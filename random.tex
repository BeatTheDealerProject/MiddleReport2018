\section{シミュレータの擬似乱数の検証}
今回シミュレータを作成するにあたり、擬似乱数を使用した。この擬似乱数が適切かどうかについて検証する。今回使用した擬似乱数生成方法はPython3のrandom関数である。このrandom関数の擬似乱数を生成するアルゴリズムはメルセンヌツイスタを用いている。そのため今回はメルセンヌツイスタについて検証する。
\bunseki{※柿崎大輝}
\subsection{周期}
擬似乱数には周期が存在する。周期とは同じ数列が出てくるようになるまでの数字の出現回数のことを指す。周期が小さいとよく同じ数列が出てきてしまいランダム性が低い。つまり周期が大きいとランダム性が高いので、性能が良いということになる。メルセンヌツイスタでは周期は$2^{19937}-1$である。これはほかの擬似乱数に比べ、かなり大きい周期である。そのため、メルセンヌツイスタを擬似乱数として使うのに十分であると考えられる。
\bunseki{※柿崎大輝}
\subsection{独立性の検定による検証}
次はカイ2乗検定を使い、どの値も等しい確率で出ていることを検証する。random関数を使用して、0~1の範囲
の乱数を生成する。その後、その値を0~0.1、0.1~0.2、0.2~0.3、0.3~0.4、0.4~0.5、0.5~0.6、0.6~0.7、0.7~0.8、0.8~0.9、0.9~1.0の10通りに分類する。それをまとめると表1.7になる。

\begin{table}[H]
 \caption{random関数での結果}
 \begin{center}
\caption{randomでの結果}
  \begin{tabular}{|c|c|c|c|c|c|c|c|c|c|} \hline 
  0~0.1 &  0.1~0.2 & 0.2~0.3 & 0.3~0.4 &  0.4~0.5 & 0.5~0.6 & 0.6~0.7 & 0.7~0.8 & 0.8~0.9 & 0.9~1.0 \\ \hline 
  95 & 85 & 100 & 102 & 91 & 114 & 87 & 108 & 115 & 103 \\ \hline
  \end{tabular}
   \label{x2}
 \end{center}
\end{table}

完全にランダムなのであれば、この結果はどれも均等に100となることが予想できる。しかし、実際は全てが100にはならずばらつきが出てしまうので、カイ2乗検定を行い検証する。先ほど出た度数を実現度数として使用し、100を理論度数としてカイ2乗検定を行う。この時、自由度は9で優位水準を5%とすると、棄却値は16.92となり、カイ2乗値がこれより小さいと擬似乱数が等しく出てきたといえる。実際に計算すると、カイ2乗値は9.98となった。この値は16.92より小さいので、擬似乱数によって出た値はすべて等しい確率であると言える。
\buseki{※薩田凱斗}
\subsection{擬似乱数まとめ}
random関数ではメルセンヌツイスタが使われ、かつ周期もカイ2乗検定においても十二分に使えると判断することができるため、今回のシミュレータにおいて使用した。
\bunseki{※柿崎大輝}
