\section{ルール}
\subsection{ブラックジャック}
ブラックジャックとは、トランプを使用したゲームで、カジノで行われる有名なギャンブルの一つである。ブラックジャックはプレイヤーとディーラー(カジノ側の
プレイヤー)が戦うゲームで、勝敗はトランプの合計で決まり、合計が21以下で相手より大きい方の勝利である。また、ブラックジャックを極めることが出来れば全
てのカジノで勝ち易くなると言われており、カジノでの勝率を上げる手段としても注目されている。

\subsection{ブラックジャックのルール}
トランプの扱いについて
\begin{itemize}
\item ジョーカーは扱わない
\item トランプのマークは関係がない
\item 数字は2~9まではそのままの数
\item 10・J・Q・Kは10として扱う
\item Aは11または1で都合の良いほうとして扱う
\end{itemize}
勝敗条件
\begin{itemize}
\item 相手よりも合計が大きく、22より小さい方が勝ち
\item 22以上であるとバーストといい、バーストした側の負けとなる
\item 合計が同じ場合は引き分けとなる
\item プレイヤーとディーラーの両方がバーストの場合、ディーラーの勝ちとなる
\item 2枚の手札で合計が21になるとナチュラルブラックジャックといい、3枚以上の合計21と対決した場合、ナチュラルブラックジャックの方が勝ちになる
\end{itemize}
賭け金の扱い
\begin{itemize}
\item プレイヤーが普通に勝つと賭け金の2倍が払い戻される
\item プレイヤーがナチュラルブラックジャックで勝つと賭け金の2.5倍が払い戻される
\item ディーラーが勝つと賭け金を没収される
\end{itemize}
プレイヤーの選択肢
\begin{itemize}
\item ヒット:カードを1枚追加すること、何度でもできる
\item スタンド:カードを引かずに今のカードで勝負すること
\item サレンダー:負けを認めて、賭け金の半分をもらうことができる。最初の行動でのみ使える。
\item ダブルダウン:賭け金を2倍にし、1度だけヒットをする。最初の行動でのみ使える。
\item スプリット:最初のカード2枚が同じ数字だった場合使用可能。最初の賭け金と同じ金額を追加して、それを2つに分割して、それぞれで勝負することができる
\item インシュランス:ディーラーの表向きのカード(アップカード)が「A」の場合使える。最初の賭け金の半分を使い、ディーラーがナチュラルブラックジャックに
なればその賭け金の2倍が払い戻される。
\item イーブンマネー:自分の手札がナチュラルブラックジャックである場合に行うインシュランスのこと。このイーブンマネーの場合は元の賭金の半額をわざわざテ
ーブルに出す必要はなく、ただ単にイーブンマネーと声を出して宣言するだけでよい。宣言するとすぐその場でディーラーは元の賭金と同じ額だけ支払ってくれる。なぜ
ならディーラーにブラックジャックが完成していようがいまいが結果は必ず同じになるからだ。もしディーラーにブラックジャックが完成していた場合、インシュランス
として掛けた保険料の倍の金額が支払われ、もともとの勝負の方はお互いブラックジャックなので引き分け。結果として保険金だけを受け取ることになる。逆にもしディ
ーラーにブラックジャックができていなかった場合、当然保険料は没収されるが、一方ゲームそのものの勝負はプレイヤー側の勝ちとなり賭金の1.5倍の勝ち金を受け
ることになる。つまり結果として差し引き 「賭金と同じ額だけの儲け」 ということになる。以上のようにイーブンマネー保険を掛けた場合はディーラーの見えていない
方のカードに関係なく自動的に賭金の1倍の勝ちが確定することになる。よってイーブンマネーの場合はイーブンマネーと宣言するだけで良い。
\end{itemize}

\subsection{ディーラーの行動}
プレイヤーに比べ、ディーラーが選択できる行動は少なく、ヒットとスタンドしかできない。その上、ディーラーの行動はあるルールが存在する。そのルールとは「ディー
ラーは17以上になるまでヒットを続けなければならない」というルールだ。これによりディーラーの最終合計は17,18,19,20,21,バーストのどれかになる。

\subsection{ゲームの流れ}
まず山札をシャッフルし、カットカードと呼ばれるカードを山札にランダムに入れる。カーットカードは、ゲームが終わりを示すカードである。その後ゲームが始まり、ま
ず賭け金を出す。その後、プレイヤー、ディーラーの順にカードが1枚配られ、再度同じように2枚目が配られます。このときプレイヤーのカードは2枚とも表向きですが
ディーラーは1枚を表向き(アップカード)でもう1枚は裏向き(ダウンカード)です。カードが配り終えると、プレイヤーの行動です。プレイヤーがバースト、もしくは
スタンドした場合、プレイヤーの行動は終了である。次はディーラーの行動です。ディーラーがスタンド、もしくはバーストした場合、ディーラーの行動は終了である。こ
こで勝敗を確認し、それに応じた支払いが行われて、ゲームが終了する。