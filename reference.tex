\begin{thebibliography}{9}
<<<<<<< HEAD
  \bibitem{pattern1} Bishop,C.M.(2007) {\it{Pattern Recognition and Machine Learning}} (元田浩 他 訳 (2016) 『パターン認識と機械学習 上』 , 丸善出版)
  \bibitem{basicstrategy} Thorp,E.(1962) {\it{Beat The Dealer:A Winning Strategy for the Game of Twenty One}} (宮崎三瑛(2006)『ディーラーをやっつけろ!』,Vintage)
  \bibitem{complexity} Chaitin,G.J(1969)"On the Simplicity and Speed of Programs for Computing Infinite Sets of Natural Numbers", {\it{Journal of the Association for Computing Machinery}}, Vol.16,No.3,July 1969,pp. 407-422
  \bibitem{blakjack2} Baldwin,R. and Cantey,W. and Maisel,H. and McDermott,J.(1956) "The Optimum Strategy in Blackjack", {\it{Journal of the American Statistical Association}}, 51:275, 419-439
=======
  \bibitem{pattern1} Christopher M. Bishop(2007) {\it{Pattern Recognition and Machine Learning}} (元田浩,栗田多喜夫,樋口知之,松本裕治,村田昇 (2016) 『パターン認識と機械学習 上』,丸善出版)
  \bibitem{basicstrategy} Edward Thorp (1962) {\it{Beat The Dealer:A Winning Strategy for the Game of Twenty One}} (宮崎三瑛(2006)『ディーラーをやっつけろ!』,Vintage)
  \bibitem{complexity} Gregory J. Chaitin(1969)"On the Simplicity and Speed of Programs for Computing Infinite Sets of Natural Numbers", {\it{Journal of the Association for Computing Machinery}}, Vol.16,No.3,July 1969,pp. 407-422
  \bibitem{blakjack2} Roger Baldwin, Wilbert Cantey, Herbert Maisel and James McDermott (1956) "The Optimum Strategy in Blackjack", {\it{Journal of the American Statistical Association}}, 51:275, 419-439
>>>>>>> 5d4c4b81d53bdd952942ea86b4dfa94f425a70f3
  \bibitem{neuro2} 斎藤康毅 (2016) 『ゼロから作るDeep Learning Pythonで学ぶディープラーニングの理論と実装』, 森北出版株式会社
  \bibitem{blackjack1} 齋藤隆浩 (1999) 『新訂ブラックジャック必勝法』, 株式会社データハウス
  \bibitem{neuro1} 萩原将文 (1994) 『ニューロ・ファジィ・遺伝的アルゴリズム』, 産業図書株式会社
  \bibitem{statistics1} 山内光哉(1987) 『心理・教育のための統計法』, 株式会社サイエンス社
\end{thebibliography}
