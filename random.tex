\section{シミュレータの擬似乱数の検証}
今回シミュレータを作成するにあたり、擬似乱数を使用した。この擬似乱数が妥当かどうかについて検証する。今回使用した擬似乱数生成方法はPythonのrandom()である。このrandom()の擬似乱数を生成するアルゴリズムはメルセンヌツイスタを用いている。そのため今回メルセンヌツイスタについて検証する。
\bunseki{※柿崎大輝}
\subsection{周期}
擬似乱数には周期が存在する。周期とは同じ数字が出てくるようになるまでの回数のことを指す。周期が小さいとよく同じ数字が出てきてしまいランダムと言いずらい。逆に周期が大きいと同じ数字が出てきずらくなる。そのため、擬似乱数では周期が大きいほうが性能が良いということになる。メルセンヌツイスタでは周期は$2^{19937}-1$である。これはほかの擬似乱数に比べ、かなり大きい周期であり、メルセンヌツイスタを擬似乱数として使うのに十分であると考えられる。
\bunseki{※柿崎大輝}
\subsection{カイ2乗検定}
次はカイ2乗検定を使い、どの値も等しい確率で出ていることを検証する。random()を使用して、0~1の範囲
で1000回値を出す。その後、その値を0~0.1、0.1~0.2、0.2~0.3、0.3~0.4、0.4~0.5、0.5~0.6、0.6~0.7、0.7~0.8、0.8~0.9、0.9~1.0の10通りに分類する。それをまとめると以下の表になる。
\begin{table}[H]
 \begin{center}
  \begin{tabular}{|c|c|c|c|c|c|c|c|c|c|}
    \hline    0~0.1 &  0.1~0.2 & 0.2~0.3 & 0.3~0.4 &  0.4~0.5 & 0.5~0.6 & 0.6~0.7 & 0.7~0.8 & 0.8~0.9 & 0.9~1.0 \\
    \hline 95 & 85 & 100 & 102 & 91 & 114 & 87 & 108 & 115 & 103 \\
    \hline
  \end{tabular}
 \end{center}
 \caption{randomでの結果}
\end{table}
本当にランダムなのであれば、この結果はどれも100になることが予想できる。しかし、実際にはすべてがピッタリ100にはならないので、そのためカイ2乗検定を行い検証する。先ほど出た度数を実現度数として使用し、100を理論度数としてカイ2乗検定を行う。この時、自由度は9で優位水準を5%とすると、棄却値は16.92となり、カイ2乗値がこれより小さいと擬似乱数がどれを等しく出てきたといえる。実際に計算すると、カイ2乗値は9.98となった。この値は16.92より小さいので、擬似乱数によって出た値はすべて等しく出てきたといえる。
\bunseki{※柿崎大輝}
\subsection{擬似乱数まとめ}
random()ではメルセンヌツイスタが使われ、かつ周期もカイ2乗検定においても十二分に使えると判断することができるため、今回のシミュレータにおいて使用した。
\bunseki{※柿崎大輝}