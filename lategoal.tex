後期の目標としては次が挙げられる。

\begin{itemize}
\item ディーラー側の行動の検証
\item デックが有限個の場合での戦略
\item 賭け金の概念の導入
\item より扱いやすい戦略の検証
\end{itemize}

ディーラー側の行動の検証については、ニューラルネットワークを用いて、プレイヤーがどのような戦略を取っているのかを検知するプログラムを作成することを目標にしている。例えば、プレイヤーがカウンティング戦略を使用している際にそれを見抜く事ができるようなプログラムを作成し、そうして作成したプログラムを用いて、ディーラー側に検知されにくい戦略の生成に活用しようと考えている。

デックが有限個の場合での戦略については、実際の対戦に従って、デック数が有限でシャッフルを一定ゲーム数まで行わず、連続でゲームを行う場合の最適な戦略を検証することを目標にしている。前述したとおりベーシックストラテジーはデック数が無限であるという前提のもと成立している戦略であるが、実際のゲームにおいてはデック数は有限であり、シャッフルを一定ゲーム数まで行わない。この事から、ゲームの進行状況により、最適とされる戦略が変わる可能性がある。そのため、デック数が有限で連続してゲームを行うと設定した状態での最適な戦略について検証することを考えている。

賭け金の概念の導入では実際のブラックジャックのゲームに則って賭け金を設定し、利得をどの様にプラスにしていくか、そのための最適な行動を考える。前期の活動では賭け金の概念は考えず、戦略の勝率のみに着目していた。しかし、実際のブラックジャックのゲームにおいては戦略の勝率が低かったとしても賭け金の賭け方によっては利得をプラスにすることが可能である。この事から、戦略の勝率のみに着目するのではなく、賭け金の賭け方にも着目し、最終的な利得をプラスにしていく戦略を検証することを考えている。

より扱いやすい戦略の検証について、今回は複雑性の設定を手動で行い、検証する時間もあまり取らなかったため、評価基準が正確ではない可能性がある。今後、この評価基準をどのように調整するかも検討しようと考えている。

\bunseki{※渡邊凛}