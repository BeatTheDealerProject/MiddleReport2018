後期の目標としては次が挙げられる。

\begin{itemize}
\item ディーラー側の行動の検証
\item デックが有限個の場合での戦略
\item 賭け金の概念の導入
\end{itemize}

ディーラー側の行動の検証ではニューラルネットワークを用いて、プレイヤーがどのような戦略を取っているのかを検知するプログラムを作成することを目標にしている。例えば、プレイヤーがカウンティング戦略を使用している際にそれを見抜く事ができるようなプログラムを作成し、そうして作成したプログラムを用いて、ディーラー側に検知されにくい戦略の生成に活用しようと考えている。

デックが有限個の場合での戦略については、実際の対戦に従って、デックが有限個の場合の最適な戦略を検証することを目標にしている。前述したとおりベーシックストラテジーはデック数が無限であるという前提のもと成立している戦略であるが、実際のゲームにおいてはデック数は有限である。この事からデック数を有限と設定した状態での最適な戦略について検証することを考えている。

賭け金の概念の導入では実際のブラックジャックのゲームに則って賭け金を設定し、利得をどの様にプラスにしていくか、そのための最適な行動を考える。前期の活動では賭け金の概念は考えず、戦略の勝率のみに着目していた。しかし、実際のブラックジャックのゲームにおいては戦略の勝率が低かったとしても賭け金の賭け方によっては利得をプラスにすることが可能である。この事から、戦略の勝率のみに着目するのではなく、賭け金の賭け方にも着目し、最終的な利得をプラスにしていく戦略を検証することを考えている。

\bunseki{※尾崎拓海}